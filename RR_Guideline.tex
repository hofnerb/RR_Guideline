\documentclass[12pt,a4paper]{article}

\usepackage{amsfonts}
\usepackage{amsmath}
\usepackage[authoryear]{natbib}
\usepackage[dvips]{epsfig}
\usepackage{dcolumn}
\usepackage{enumerate}
\usepackage{hhline}
\usepackage{dsfont}
\usepackage{textcomp}
\usepackage{lineno}
\usepackage{graphicx}
\usepackage{color}
\usepackage{rotating}
\usepackage[colorlinks=TRUE,urlcolor=blue,citecolor=blue,linkcolor=black]{hyperref}
\usepackage{multirow}
\usepackage{natbib}
\usepackage{fancyvrb}

\usepackage{titlesec}

\titleformat*{\section}{\large\bfseries}
\titleformat*{\subsection}{\large\bfseries}
\titleformat*{\subsubsection}{\bfseries}

\sloppy
\setlength{\parindent}{0cm}
\setlength{\parskip}{0em}

\setlength{\paperheight}{29.7cm}
\setlength{\paperwidth}{21cm}

\setlength{\textheight}{23cm}
\setlength{\textwidth}{15.5cm}

\setlength{\headheight}{0cm}
\setlength{\headsep}{0cm}
\setlength{\topskip}{0cm}

\setlength{\topmargin}{0cm}
\setlength{\oddsidemargin}{0cm}

\makeatletter
\renewcommand{\maketitle}{\bgroup\setlength{\parindent}{0pt}
  \begin{flushleft}
    \vspace*{0cm}
    \textbf{\Large \@title}\\[1em]

    \@author\\[0.5em]

    \textbf{Document Version:} \@date
    \vspace{1em}
  \end{flushleft}\egroup
}
\makeatother

\title{\LARGE Guidelines for Code and Data Submission\\[0.2em]
  \large Specific Guidance on Reproducible Research (RR)}

\author{Benjamin Hofner (RR Editor, Biometrical Journal)\\
  E-mail: \href{mailto:benjamin.hofner@fau.de}{benjamin.hofner@fau.de}}

\date{1.7 (2016/xx/yy)}

\begin{document}

\maketitle

\hrule
\vspace{-0.7em}
\tableofcontents

\vspace{1em}
\hrule
\vspace{0.5cm}

\setlength{\parskip}{0.2em}

\section*{Why do we support reproducible research (RR)?}
\begin{itemize}
\item Code (and data) makes it easier (or even possible) to reproduce the results
  and thus facilitates understanding of the methods and results.
\item Publicly available code helps to promote the results and methods of the
  article and increases scientific impact.
\item Good code increases the precision of the description of methods and results
  and thus improves the article.
\item Code facilitates communication with other researches who intend to apply
  the proposed methods or develop alternatives to them.
\end{itemize}

For more information on reproducible research, a review of the RR policy of the
Biometrical Journal and its impact, and on the RR guideline with background
information we refer to Hofner \emph{et al.} \cite{Hofner:2015}.


\newpage
\setcounter{section}{0}

\section{Submission material}

Please prepare your (code) submission as follows:

\subsection{Code and data}

\begin{itemize}
\item The code must be well documented. Each user-defined function
  (\textsf{Stata}: program) must be documented including all function
  arguments (\textsf{Stata}: options).
\item Unnecessary comments (such as alternative code lines) {should} be
  avoided.
\item Please use a consistent style throughout your code.
  \begin{itemize}
  \item \textsf{R}: For a good example see, e.g.,
    \url{http://adv-r.had.co.nz/Style.html}.
  \end{itemize}
 \item Please indent your code properly. Usually 2 or 4 spaces per indentation
  level are a good choice. Maintain a consistent indentation style throughout
  your document.
  \begin{itemize}
  \item \textsf{R}: If you use RStudio
    (\href{http://www.rstudio.com/products/rstudio/download/}{Download}), proper
    indentation can be easily achieved by selecting everything (\texttt{CTRL-A})
    and then typing \texttt{CTRL-I} (re-indent lines).
  \item \textsf{Stata}: use tab-indentation (which defaults to 4 spaces in the
    \textsf{Stata} do-file editor).
  \end{itemize}
\item Try to put the code in as few code files as possible but use separate
  files where useful and sensible, e.g., one for the function definitions, one
  for the simulation study and another for the data analysis.
  \begin{itemize}
  \item \textsf{R}, \textsf{Stata}, etc.: Do not use separate files for each function,
    but do structure the code files properly using comments for each "section".
  \item \textsf{MATLAB}: use one file per function.
  \end{itemize}
\item Read other files, such as the function definitions, into the main file performing the analysis, instead of mixing code that performs the analysis with code that defines the functions used in the analysis:
  \begin{itemize}
  \item \textsf{R}: use \texttt{source()}.
  \item \textsf{Stata}: use \texttt{do}.
  \item \textsf{SAS}: use \texttt{\%include}.
  \end{itemize}
\item Make sure that the results are easy to compare with the manuscript:
  \begin{itemize}
  \item Add comments that state what figure / table in the manuscript the code produces.
  \item Structure your output such that it corresponds to the structure of the
    tables given in the manuscript.
  \end{itemize}
\item Avoid absolute paths (e.g., \verb+C:/My Code+) whenever possible. Use
  paths relative to the current file instead. If this is impossible state in
  your README where/how to change the paths.
  \begin{itemize}
  \item In \textsf{R} and \textsf{Stata} absolute paths are not necessary
    (unless in some very rare occasions). If you really want to set or change
    your working directory do this only once, at the very beginning of your
    master code file. For an example that uses relative paths in \textsf{R} see
    Appendix~\ref{sec:using-relative-paths}.
  \item In \textsf{SAS} one can either define a macro variable that contains the
    relevant path and use this wherever the path is needed or one needs to
    clearly state in the README which path needs to be replaced in which files.
  \end{itemize}
\item Use descriptive file names and avoid spaces in file names. Use underscores
  (\_) or hyphens (-) instead.
\item Each submission {should} contain:
\begin{itemize}
\item all the code to reproduce all the results, images and tables. Make sure
  that the code submitted is complete and covers all calculations used in the
  finally published paper (and only these). Make also sure the code is well
  documented. This especially includes (but is not restricted to) the
  documentation of user written functions and arguments therein.
\item all the data to reproduce all the results, images and tables. % Original
  % data sets and especially huge data sets can also be uploaded to specialized
  % research data repositories, which often assign DOIs or other persistent
  % identifiers to the data sets and thus make them citable. For available
  % repositories see the Registry of Research Data Repositories
  % (\url{http://www.re3data.org}). A prominent
  % multidisciplinary repository is Figshare (\url{http://figshare.com}).
\end{itemize}
\item If the code takes a long time to run  on a modern PC (longer than 2h, say),
  consider saving intermediate results in a separate folder. Keep the main folder
  clean so that the code is executed without using the intermediate results by default.
  For an example of the structure, see Section~\ref{sec:structure}. Please state in the README
  how to use the intermediate results (e.g. ``Copy the intermediate results to
  the folder ...''). Alternatively, give advice how to change certain parameters
  (such as the number of repetitions) to speed up the computation drastically
  but still obtains similar results.
\item If the data is confidential, authors should consider using simulated data
  that mimics (the main features of) the real data and produces similar results.
\item Simulations and everything else that relies on random numbers / random
  samples {must} set a random generator seed to make the results reproducible:
  \begin{itemize}
  \item \textsf{R}: use \texttt{set.seed(\#)}
  \item \textsf{Stata}: use \texttt{set seed \#}
  \item \textsf{SAS}: use for example \texttt{call streaminit(\#)}, or
    \texttt{call randseed(\#)} in \texttt{proc iml}
  \end{itemize}
  (replace \texttt{\#} by a number)
\item Make sure that the code can be \emph{executed without the need of
    alterations by the user}. This avoids user errors and streamlines the
  reproduction of results for both the reviewers and readers. If user input is
  required, the authors must state and explain this in the README.
\item Before submitting the code carefully check that it reproduces the published
  results. Especially make sure that it is executable in a clean workspace. In
  \textsf{R}, remove all objects from the environment via \texttt{rm(list =
    ls())} before running the code or use \texttt{R CMD BATCH} to start the code.
\end{itemize}

\subsection{README}

\begin{itemize}
\item Each code submission {must} be accompanied by a \texttt{README.txt} or a
  \texttt{README.pdf} file that contains:
  \begin{itemize}
  \item the title of the manuscript
  \item the authors of the manuscript
  \item please indicate which authors are mainly responsible for writing the
    code in the README. Please name any other persons that made substantial
    contributions to the development of the code here and also acknowledge them
    in the paper.
  \item the e-mail address of the author or contributor that readers should address with
    questions, comments and remarks on the code (and to report bugs).
  \item a list of configurations on which you tested your code [software,
    software version (incl. package versions), platform].
    \begin{itemize}
    \item In \textsf{R} you can use the output from \texttt{sessionInfo()} at
      the end of your code, i.e., after all packages have been loaded.
    \item In \textsf{Stata} report the version number and if appropriate the
      flavor you are using (see \texttt{Help > About Stata}).
    \end{itemize}
  \item a specification of special hardware or software requirements (if
    applicable).
  \end{itemize}
  \item The README
  \begin{itemize}
  \item explains how to execute the code (e.g., Which program was used? In which
    order should the files be executed? What are the results obtained by the
    file? Add flowcharts when they appear helpful.)
  \item (if applicable) explains what the data represents, i.e., the type and
    meaning of all data which are used in the code. You may refer to
    tables/sections in the paper.
  \item state the source of the data.
  \end{itemize}
\end{itemize}

\subsection{Structure of code submission (ZIP-folder)}\label{sec:structure}

Code, data and the README {must} be contained in a single zip container,
possibly with sub-folders in the container for different purposes. E.g.\ one
could have the following structure

\begin{verbatim}
README.txt
/case_study
    ./data
    case_study.R
/simulation
    ./intermediate_results
    simulation.R
\end{verbatim}

A special folder for the code might not be necessary. The README should be
placed in the top level of the zip file (see example above). Make sure that the
code can be directly executed after unpacking the file without any need to alter
the code.

Please name the zip container as follows:
\begin{itemize}
\item If it contains only data: \texttt{Data.zip}
\item If it contains only Code: \texttt{Code.zip}
\item If it contains data and code: \texttt{Code\_and\_Data.zip}
\end{itemize}

\newpage

\section{Submission process}

Please use the appropriate mode of submission:

\begin{itemize}
\item \textbf{When you submit your paper:}\\[0.5em]
  If your paper fits the RR option of Biometrical Journal you are
  welcome to submit code and data along with the first submission of the
  manuscript. Please upload the zip file, structured as described in Section
  \ref{sec:structure}, on Manuscript Central via "Data and Software".
\item \textbf{During the review process:}\\[0.5em]
  You may be asked during the review process to submit code and data or to
  change them if they were already supplied at the time of the initial submission.
  Then please upload the zip file, structured as described in Section
  \ref{sec:structure}, on Manuscript Central via "Data and Software".
\item \textbf{After (conditional) acceptance:}\\[0.5em]
  Please send the code (structured as described in Section \ref{sec:structure}) to the RR Editor
  (\href{mailto:benjamin.hofner@fau.de}{benjamin.hofner@fau.de}). If the
  manuscript was changed please send also the manuscript (incl. sources) by
  email. If the files are too large to be emailed please inform the RR Editor
  who will give further instructions on how to proceed. All files will be
  uploaded to Manuscript Central at the end of the RR check.
\end{itemize}

If any questions on the submission process arise please contact the RR Editor
and keep the Editor in cc.

% There should be new last section for contacting persons which should contain the
% e-mail for direct communication with the RR-Editor, but also the recommendation
% to communicate within ScholarOne with the RR-editor where admin ( Monika) is
% automatic in cc. As long as this option is not implmented we can only say that
% authors can contact the RR-editor directly.

\section{Reference the code}

Please refer to the code in the submitted manuscript where appropriate, for
example in the section of the data analysis and/or in the section where the
simulation study or a numerical study are described. You can do this for example
as follows:
\begin{quote}
  ``Source code to reproduce the results is available as Supporting Information
  on the journal's web page (\url{http://onlinelibrary.wiley.com/doi/xxx/suppinfo}).''
\end{quote}

If the Supporting Information contains several files you may list them
explicitly such that the reader (of the electronic version) can navigate through
the Supporting Information. The links will be set by the publisher.

\section{Example}

A good example is given by W. Sauerbrei, A. Buchholz, A.-L. Boulesteix \& H.
Binder (see \url{http://onlinelibrary.wiley.com/enhanced/doi/10.1002/bimj.201300222/}).


\section{Possible improvements}

To further simplify and enhance reproducibility of results, submissions using
\href{http://en.wikipedia.org/wiki/Literate_Programming}{literate programming}
approaches to combine \LaTeX and \textsf{R} code such as
\href{http://en.wikipedia.org/wiki/Sweave}{Sweave}
\citep{Leisch:2002,Leisch:2003}/
\href{http://en.wikipedia.org/wiki/Knitr}{knitr} \citep{Xie:2013,Xie:2014}/\ldots
are encouraged. Please also supply an \textsf{R} code file in that case. For
\textsf{SAS} and \textsf{Stata} there exists a similar package for dynamic
documents called
\href{http://homepage.stat.uiowa.edu/~rlenth/StatWeave/}{StatWeave}.

Authors of papers that use multiple user-specified functions coded in \textsf{R}
are encouraged to provide an \textsf{R} package (which can be, but does not
necessarily has to be, hosted on \href{http://cran.r-project.org/}{CRAN} or
\href{http://bioconductor.org/}{Bioconductor}).

%\clearpage

\bibliographystyle{plain}
\bibliography{bib}

\clearpage
\appendix
\section{Using relative paths in R}
\label{sec:using-relative-paths}

Normally, you can assume that the working directory is the folder that contains
the master code file. Now, one can navigate relative to this directory. Assume
the following (unnecessarily complex) structure:

\begin{verbatim}
/data/dataset.Rda
/code/case_study.R
/code/functions/fun_dev.R
/results/
\end{verbatim}

We assume that the working directory is the folder of the master source
file, i.e., \texttt{code}. Now, we want to load the function definitions
\texttt{fun\_def.R} into the master file \texttt{case\_study.R}, load the
data set and save the results in the \texttt{results} folder. The code in
\texttt{case\_study.R} looks as follows:

\begin{Verbatim}[frame=single]
######################################################################
## This an example with pseudo code to show the use of relative paths
######################################################################

## make sure the current working directory is the folder code/

## now source the function definitions:
source("functions/fun_dev.R")

## load the data
load("../data/dataset.Rda")
## ../ means to go up one directory level,
## ../../ would mean to go up two levels etc.

## do something with the data

## save the results to results/...
pdf("../results/figure1.pdf")
plot(...)
dev.off()

write.csv(some_results, file = "../results/table1.csv")
\end{Verbatim}


\end{document}

