\documentclass[12pt,a4paper]{article}

\usepackage{amsfonts}
\usepackage{amsmath}
\usepackage[authoryear]{natbib}
\usepackage[dvips]{epsfig}
\usepackage{dcolumn}
\usepackage{enumerate}
\usepackage{hhline}
\usepackage{dsfont}
\usepackage{textcomp}
\usepackage{lineno}
\usepackage{graphicx}
\usepackage{color}
\usepackage{rotating}
\usepackage[colorlinks=TRUE,urlcolor=blue,citecolor=blue,linkcolor=black]{hyperref}
\usepackage{multirow}
\usepackage{natbib}

\usepackage{titlesec}

\titleformat*{\section}{\large\bfseries}
\titleformat*{\subsection}{\large\bfseries}
\titleformat*{\subsubsection}{\bfseries}

\sloppy
\setlength{\parindent}{0cm}
\setlength{\parskip}{0em}

\setlength{\paperheight}{29.7cm}
\setlength{\paperwidth}{21cm}

\setlength{\textheight}{23cm}
\setlength{\textwidth}{15.5cm}

\setlength{\headheight}{0cm}
\setlength{\headsep}{0cm}
\setlength{\topskip}{0cm}

\setlength{\topmargin}{0cm}
\setlength{\oddsidemargin}{0cm}

\makeatletter
\renewcommand{\maketitle}{\bgroup\setlength{\parindent}{0pt}
  \begin{flushleft}
    \vspace*{3cm}
    \textbf{\Large \@title}\\[1em]

    \@author\\[0.5em]

    \textbf{Document Version:} \@date
    \vspace{2em}
  \end{flushleft}\egroup
}
\makeatother

\title{Specific Guidance on Reproducible Research (RR)\\[0.2em]
  \large Guideline for Code Submissions}

\author{Benjamin Hofner (RR Editor, Biometrical Journal)\\
  E-mail: \href{mailto:benjamin.hofner@fau.de}{benjamin.hofner@fau.de}}

\date{1.4 (2015/04/xx)}

\begin{document}

\maketitle

\hrule

\tableofcontents

\vspace{2em}
\hrule
\vspace{0.5cm}

\setlength{\parskip}{0.2em}

\section*{Why do we support reproducible research (RR)?}
\begin{itemize}
\item Code (and data) makes it easier (or even possible) to reproduce the results
  and thus facilitates understanding of the methods and results.
\item Code helps to promote the results and methods of the article.
\item Good code adds to the precision of the description of methods and results
  and thus improves the article.
\item Code facilitates communication with other researches who intend to apply
  the proposed methods.
\end{itemize}

\newpage
\setcounter{section}{0}

\section{Submission Material}

Please prepare your (code) submission as follows:

\subsection{Code and Data}

\begin{itemize}
\item The code {must} be well documented. Each user-defined function
  (\textsf{Stata}: program) {must} be documented including all function
  arguments (\textsf{Stata}: options).
\item Unnecessary comments (such as alternative code lines) {should} be
  avoided.
\item Please indent for your code properly. Usually 2 or 4 spaces per indention
  level are a good choice. Keep that indention style throughout your document.
  \begin{itemize}
  \item \textsf{R}: If you use RStudio
    (\href{http://www.rstudio.com/products/rstudio/download/}{Download}), proper
    indention can be easily achieved by selecting everything (\texttt{CRTL-A})
    and then typing \texttt{CTRL-I} (re-indent lines).
  \item \textsf{Stata}: use tab-indention (which defaults to 4 spaces in the
    \textsf{Stata} do-file editor).
  \end{itemize}
\item Try to put the code in as few code files as possible but use separate
  files where useful and sensible, e.g., one for the function definitions, one
  for the simulation study and another for the data analysis.
  \begin{itemize}
  \item \textsf{R}, \textsf{Stata}, etc.: Do not use separate files for each function,
    but do structure the code files properly using comments for each "section".
  \item \textsf{MATLAB}: use one file per function.
  \end{itemize}
\item Read other files, such as the function definitions, into the main file:
  \begin{itemize}
  \item \textsf{R}: use \texttt{source()}.
  \item \textsf{Stata}: use \texttt{do}.
  \item \textsf{SAS}: use \texttt{\%include}.
  \end{itemize}
\item Make sure that the results are easy to compare with the manuscript:
  \begin{itemize}
  \item Add comments that state what figure / table the code produces.
  \item Structure your output such that it corresponds to the structure of the
    table given in the manuscript.
  \end{itemize}
\item Avoid absolute paths (e.g., \verb+C:/My Code+) whenever possible. Use
  paths relative to the current file instead. If this is impossible state in
  your README where/how to change the paths.
  \begin{itemize}
  \item In \textsf{R} and \textsf{Stata} absolute paths are not necessary
    (unless in some very rare occasions). If you really want to set or change
    your working directory do this only once, at the very beginning of your
    master code file.
  \item In \textsf{SAS} one can either define a macro variable that contains the
    relevant path and use this wherever the path is needed or one needs to
    clearly state in the README which path needs to be replaced in which files.
  \end{itemize}
\item Avoid spaces in file names. Use underscores (\_) or hyphens (-) instead.
\item Use descriptive file names.
\item Each submission {should} contain:
\begin{itemize}
\item all the code to reproduce all the results, images and tables. Make sure
  that the code submitted is complete and covers all calculuations used in the
  finally published paper (and only these). Make also sure the code is well
  documented. This especially includes (but is not restricted to) the
  documentation of user written functions and arguments therein.
\item all the data to reproduce all the results, images and tables.
\end{itemize}
\item If the code needs more time to run (say longer than 2h), consider adding
  intermediate results in a separate folder. Keep the main folder clean so that
  usually the code is run without using the intermediate results. For an example
  of the structure see Section~\ref{sec:structure}. Please state in the README
  how to use the intermediate results (e.g. ``Copy the intermediate results to
  the folder ...'').
\item If the data is confidential, authors should consider using simulated data
  that mimics (the main features of) the real data and produces similar results.
\item Simulations and everything else that relies on random numbers / random
  samples {must} use a seed to make the results reproducible (in \textsf{R} use
  \texttt{set.seed(...)}; in \textsf{Stata} use \texttt{set seed \#}).
\item Make sure that the code can be executed without the need of alterations by
  the user. If user input is really needed, please state and explain this in the
  README.
\item Before submitting the code make sure that it is executable in a fresh
  environment. In \textsf{R} first remove all objects from the environment via
  \texttt{rm(list = ls())} or use \texttt{R CMD BATCH} to start the code.
\end{itemize}

\subsection{README}

\begin{itemize}
\item Each code submission {must} be accompanied by a \texttt{README.txt} or a
  \texttt{README.pdf} file that must contain:
  \begin{itemize}
  \item the title of the manuscript
  \item the authors of the manuscript
  \item if one or more than one of the authors were the main contributors/producers
    of the code name them. If other persons collaborated substantially in the
    establishment of the code and are not authors of the paper name them and
    achnowledge them in the paper.
  \item one e-mail address such that readers can approach that person for
    questions, comments and remarks (and to report bugs).
  \item a list of configurations on which you tested your code [software,
    software version (incl. package versions), platform].
    \begin{itemize}
    \item In \textsf{R} you can use the output from \texttt{sessionInfo()} at
      the end of your code, i.e., after all packages have been loaded.
    \item In \textsf{Stata} report the version number and if appropriate the
      flavour you are using (see \texttt{Help > About Stata}).
    \end{itemize}
  \item a specification of special hardware or software requirements (if
    applicable).
  \end{itemize}
  \item The README
  \begin{itemize}
  \item explains how to execute the code (e.g., Which program was used? In which
    order should the files be executed? What are the results obtained by the
    file? Add flowcharts when they appear helpful.)
  \item (if applicable) explains what the data represents, i.e., the type and
    meaning of all data which are used in the code. You may refer to
    tables/sections in the paper.
  \item state the source of the data.
  \end{itemize}
\end{itemize}

\subsection{Structure of Submission (ZIP-folder)}\label{sec:structure}

Code, data and the README {must} be contained in a single zip container,
possibly with sub-folders in the container for different purposes. E.g.\ one
could have the following structure

\begin{verbatim}
    README.txt
    \case_study
        .\data
        case_study.R
    \simulation
        .\intermediate_results
        simulation.R
\end{verbatim}

A special folder for the code might not be necessary. The README should be
placed in the top level of the zip file (see example above). Make sure that the
code can be directly executed after unpacking the file without any need to alter
the code.

\newpage

\section{Submission Process}

Please use the appropriate mode of submission:

\begin{itemize}
\item \textbf{When you submit your paper:}\\[0.5em]
  When your paper suits the RR option of the Biometrical Journal you are
  invited to submit code and data already at the first submission of the
  manuscript. Please upload the zip file on Manuscript Central as "Data and
  Software".
\item \textbf{During the review process:}\\[0.5em]
  You may be asked during the review process to submit code and data or to
  change them in the case of initial submission. Then please upload the zip file
  on Manuscript Central as "Data and Software".
\item \textbf{After (conditional) acceptance:}\\[0.5em]
  Please send the code to the RR Editor
  (\href{mailto:benjamin.hofner@fau.de}{benjamin.hofner@fau.de}). If the
  manuscript was changed please send also the manuscript (incl. sources) by
  email. If the files are too large to be emailed please inform the RR Editor
  who will give further instructions on how to proceed. All files will be
  uploaded to Manuscript Central at the end of the RR check.
\end{itemize}

If any questions on the submission process arise please contact the RR Editor
and keep the Editor in cc.

% There should be new last section for contacting persons which should contain the
% e-mail for direct communication with the RR-Editor, but also the recommendation
% to communicate within ScholarOne with the RR-editor where admin ( Monika) is
% automatic in cc. As long as this option is not implmented we can only say that
% authors can contact the RR-editor directly.

\section{Reference the code}

Please refer to the code in the submitted manuscript where appropriate, for
example in the section of the data analysis and/or in the section where the
simulation study or a numerical study are described. You can do this for example
as follows:
\begin{quote}
  ``Source code to reproduce the results is available as Supporting Information
  on the journal's web page (\url{http://onlinelibrary.wiley.com/doi/xxx/suppinfo}).''
\end{quote}

If the Supporting Information contains several files you may list them
explicitly such that the reader (of the electronic version) can navigate through
the Supporting Information. The links will be set by the publisher.

\section{Example}

A good example is given by W. Sauerbrei, A. Buchholz, A.-L. Boulesteix \& H.
Binder (see \url{http://onlinelibrary.wiley.com/enhanced/doi/10.1002/bimj.201300222/}).


\section{Possible improvements}

To further simplify and enhance reproducibility of results, submissions using
\href{http://en.wikipedia.org/wiki/Literate_Programming}{literate programming}
approaches to combine \LaTeX and \textsf{R} code such as
\href{http://en.wikipedia.org/wiki/Sweave}{Sweave}
\citep{Leisch:2002,Leisch:2003}/
\href{http://en.wikipedia.org/wiki/Knitr}{knitr} \citep{Xie:2013,Xie:2014}/...
are encouraged. Please also supply an \textsf{R} code file in that case. For
\textsf{SAS} and \textsf{Stata} there exists a similar package for dynamic
documents called
\href{http://homepage.stat.uiowa.edu/~rlenth/StatWeave/}{StatWeave}.

Authors of papers that use multiple user-specified functions coded in \textsf{R}
are encouraged to provide an \textsf{R} package (which can be, but does not
necessarily has to be, hosted on \href{http://cran.r-project.org/}{CRAN} or
\href{http://bioconductor.org/}{Bioconductor}).

%\clearpage

\bibliographystyle{plain}
\bibliography{bib}

\end{document}

